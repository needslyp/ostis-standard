\begin{SCn}
\scnsectionheader{\currentname}
\begin{scnsubstruct}
	\begin{scnrelfromlist}{дочерний раздел}
	%TODO: check by human--->
		\scnitem{\nameref{sd_scp_denote_sem}}
		\scnitem{\nameref{sd_scp_oper_sem}}
	%<---TODO: check by human
	\end{scnrelfromlist}
	
\scnheader{Предметная область Базового языка программирования ostis-систем (языка SCP --- Semantic Code Programming)}
\scnidtf{Предметная область Базового языка программирования ostis-систем}
\scnidtf{Предметная область Языка SCP}
\scntext{примечание}{В данную предметную область включаются все тексты программ Языка SCP. В ней исследуется типология операторов этих программ и заданные на них отношения.}
\scniselement{предметная область}
\begin{scnhaselementrole}{максимальный класс объектов исследования}
	{scp-программа}
\end{scnhaselementrole}
\begin{scnhaselementrolelist}{класс объектов исследования}
	%TODO: check by human--->
	\scnitem{агентная scp-программа}
	\scnitem{scp-процесс}
	\scnitem{scp-оператор}
	\scnitem{атомарный тип scp-оператора}
	%<---TODO: check by human
\end{scnhaselementrolelist}
\begin{scnhaselementrolelist}{исследуемое отношение}
	%TODO: check by human--->
	\scnitem{начальный оператор\scnrolesign}
	\scnitem{параметр scp-программы\scnrolesign}
	\scnitem{in-параметр\scnrolesign}
	\scnitem{out-параметр\scnrolesign}
	\scnitem{scp-операнд\scnrolesign}
	%<---TODO: check by human
\end{scnhaselementrolelist}

\scnheader{Язык SCP}
\scnidtftext{часто используемый sc-идентификатор}{scp-программа}
\scntext{пояснение}{В качестве базового языка для описания программ обработки текстов\textit{SC-кода} предлагается \textit{Язык SCP}.\\
	\textit{Язык SCP} --- это графовый язык процедурного программирования, предназначенный для эффективной обработки \textit{sc-текстов}. \textit{Язык SCP} является языком параллельного асинхронного программирования.\\
	Языком представления данных для текстов \textit{Языка SCP} (\textit{scp-программ}) является \textit{SC-код} и, соответственно, любые варианты его внешнего представления. \textit{Язык SCP} сам построен на основе \textit{SC-кода}, в следствие чего \textit{scp-программы} сами по себе могут входить в состав обрабатываемых данных для \textit{scp-программ}, в т.ч. по отношению к самим себе. Таким образом, \textit{язык SCP} предоставляет возможность построения реконфигурируемых программ. Однако для обеспечения возможности реконфигурирования программы непосредственно в процессе ее интерпретации необходимо на уровне интерпретатора \textit{Языка SCP (Aбстрактной scp-машины)} обеспечить уникальность каждой исполняемой копии исходной программы. Такую исполняемую копию, сгенерированную на основе \textit{scp-программы}, будем называть \textit{scp-процессом}. Включение знака некоторого \textit{действия в sc-памяти} во множество \textit{scp-процессов} гарантирует тот факт, что в декомпозиции данного действия будут присутствовать только знаки элементарных действий (\textit{scp-операторов}), которые может интерпретировать реализация \textit{Aбстрактной scp-машины} (интерпретатора scp-программ).\\
	\textit{Язык SCP} рассматривается как ассемблер для семантического компьютера.}

\scnheader{Базовая модель обработки sc-текстов}
\begin{scnreltoset}{объединение}
	%TODO: check by human--->
	\scnitem{Предметная область Базового языка программирования ostis-систем}
	\scnitem{Модель Абстрактной scp-машины}
	%<---TODO: check by human
\end{scnreltoset}
\begin{scnrelfromset}{особенности}
	%TODO: check by human--->
	\scnfileitem{Тексты программ \textit{Языка SCP} записываются при помощи тех же унифицированных семантических сетей, что и обрабатываемая информация, таким образом, можно сказать, что \textit{Синтаксис языка SCP} на базовом уровне совпадает с \textit{Синтаксисом SC-кода}.}
	\scnfileitem{Подход к интерпретации \textit{scp-программ} предполагает создание при каждом вызове \textit{scp-программы} уникального \textit{scp-процесса}.}
	%<---TODO: check by human
\end{scnrelfromset}
\begin{scnrelfromset}{достоинства}
	%TODO: check by human--->
	\scnfileitem{Одновременно в общей памяти могут выполняться несколько независимых\textit{sc-агентов}, при этом разные копии \textit{sc-агентов} могут выполняться на разных серверах, за счет распределенной реализации интерпретатора sc-моделей (\textit{платформы реализации sc-моделей компьютерных систем}). Более того, \textit{Язык SCP} позволяет осуществлять параллельные асинхронные вызовы подпрограмм с последующей синхронизацией, и даже параллельно выполнять операторы в рамках одной \textit{scp-программы}.}
	\scnfileitem{Перенос \textit{sc-агента} из одной системы в другую заключается в простом переносе фрагмента базы знаний, без каких-либо дополнительных операций, зависящих от платформы интерпретации.}
	\scnfileitem{Тот факт, что спецификации \textit{sc-агентов} и их программы могут быть записаны на том же языке, что и обрабатываемые знания, существенно сокращает перечень специализированных средств, предназначенных для проектирования машин обработки знаний, и упрощает их разработку за счет использования более универсальных компонентов.}
	\scnfileitem{Тот факт, что для интерпретации \textit{scp-программы} создается соответствующий ей уникальный \textit{\mbox{scp-процесс}}, позволяет по возможности оптимизировать план выполнения перед его реализацией и даже непосредственно в процессе выполнения без потенциальной опасности испортить общий универсальный алгоритм всей программы. Более того, такой подход к проектированию и интерпретации программ позволяет говорить о возможности создания самореконфигурируемых программ.}
	%<---TODO: check by human
\end{scnrelfromset}

\scnheader{Абстрактная scp-машина}
\scnrelfrom{модель}{Модель Абстрактной scp-машины}
\scntext{примечание}{\textit{Абстрактная scp-машина} представляет собой интерпретатор \textit{scp-программ}, который должен являться частью \textit{платформы интерпретации sc-моделей компьютерных систем} (хотя в общем случае могут существовать варианты платформы, не содержащие такого интерпретатора, что, однако, не позволит использовать достоинства предлагаемой базовой модели}

\scnheader{scp-программа}
\scnsubset{программа в sc-памяти}
\scntext{пояснение}{Каждая \textbf{\textit{scp-программа}} представляет собой \textit{обобщенную структуру}, описывающую один из вариантов декомпозиции действий некоторого класса, выполняемых в sc-памяти. Знак \textit{sc-переменной}, соответствующей конкретному декомпозируемому действию является в рамках \textbf{\textit{scp-программы}} \textit{ключевым sc-элементом\scnrolesign}. Также явно указывается принадлежность данного знака множеству \textit{scp-процессов}.\\
	Принадлежность некоторого действия множеству \textit{scp-процессов} гарантирует тот факт, что в декомпозиции данного действия будут присутствовать только знаки элементарных действий (\textit{scp-операторов}), которые может интерпретировать реализация абстрактной scp-машины.\\
	Таким образом, каждая \textbf{\textit{scp-программа}} описывает в обобщенном виде декомпозицию некоторого \textit{\mbox{scp-процесса}} на взаимосвязанные \textit{scp-операторы}, с указанием, при их наличии, аргументов для данного \textit{scp-процесса}.\\
	По сути каждая \textbf{\textit{scp-программа}} представляет собой описание последовательности элементарных операций, которые необходимо выполнить над семантической сетью, чтобы выполнить более сложное действие некоторого класса.}
\scnrelfrom{описание примера}{\scnfileimage[20em]{figures/sd_scp/program_example.png}}
	\begin{scnindent}
		\scntext{пояснение}{В приведенном примере показана \textit{scp-программа}, состоящая из трех \textit{scp-операторов}. Данная программа проверяет, содержится ли в заданном множестве (первый параметр) заданный элемент (второй параметр), и, если нет, то добавляет его в это множество.}
	\end{scnindent}

\scnhaselementrole{пример}{
	\scnheaderlocal{\_scp\_process}
	\scnisvarelement{scp-процесс}
	\scnhasvarelementrole{1;in-параметр}{\_set1}
	\scnhasvarelementrole{2;in-параметр}{\_ element1}
	\scnvarrelto{декомпозиция действия}{\_ ...}
	\begin{scnindent}
		\scnhasvarelementrole{1}{\_ operator1}
			\begin{scnindent}
				\scnisvarelement{searchElStr3}
				\scnhasvarelementrole{1; scp-операнд с заданным значением; scp-константа}{\_ set1}
				\scnhasvarelementrole{2; scp-операнд со свободным значением; scp-переменная; sc-дуга основного вида}{\_ arc1}
				\scnhasvarelementrole{3; scp-операнд с заданным значением; scp-константа}{\_ element1}
				\scnvarrelfrom{последовательность действий при отрицательном результате}{\_ operator2}
				\scnvarrelfrom{последовательность действий при положительном результате}{\_ operator3}
			\end{scnindent}
		\scnhasvarelement{\_ operator2}
			\begin{scnindent}
				\scnisvarelement{genElStr3}
				\scnhasvarelementrole{1:: scp-операнд с заданным значением; scp-константа}{\_ set1}
				\scnhasvarelementrole{2:: scp-операнд со свободным значением; scp-переменная; sc-дуга основного вида}{\_ arc1}
				\scnhasvarelementrole{3:: scp-операнд с заданным значением; scp-константа}{\_ element1}
				\scnvarrelfrom{следующий оператор}{\_ operator3}
			\end{scnindent}
		\scnhasvarelement{\_ operator3}
			\begin{scnindent}
				\scnisvarelement{return}
			\end{scnindent}
	\end{scnindent}
}
\end{scnsubstruct}

\scnheader{агентная scp-программа}
\scnsubset{scp-программа}
\scntext{пояснение}{\textbf{\textit{агентные scp-программы}} представляют собой частный случай \textit{scp-программ} вообще, однако заслуживают отдельного рассмотрения, поскольку используются наиболее часто. \textit{Scp-программы} данного класса представляют собой реализации программ агентов обработки знаний и имеют жестко фиксированный набор параметров. Каждая такая программа имеет ровно два \textit{in-параметра\scnrolesign}. Значение первого параметра является знаком бинарной ориентированной пары, являющейся вторым компонентом связки отношения \textit{первичное условие инициирования*} для абстрактного \textit{sc-агента}, во множество \textit{программ sc-агента*} которого входит рассматриваемая \textbf{\textit{агентная scp-программа}}, и, по сути, описывает класс событий, на которые реагирует указанный sc-агент.\\
	Значением второго параметра является \textit{sc-элемент}, с которым непосредственно связано событие, в результате возникновения которого был инициирован соответствующий \textit{sc-агент}, т.е., например, сгенерированная либо удаляемая \textit{sc-дуга} или \textit{sc-ребро}.}

\scnheader{абстрактный sc-агент, реализуемый на Языке SCP}
\begin{scnrelfromset}{принципы реализации}
	%TODO: check by human--->
	\scnfileitem{Общие принципы организации взаимодействия \textit{sc-агентов} и пользователей \textit{ostis-системы} через общую\textit{sc-память}.}
	\scnfileitem{В результате появления в sc-памяти некоторой конструкции,удовлетворяющей условию инициирования какого-либо \textit{абстрактногоsc-агента}, реализованного при помощи \textit{Языка SCP}, в \textit{sc-памяти} генерируется и инициируется \textit{scp-процесс}. В качестве шаблона для генерации используется \textit{агентная scp-программа}, указанная во множестве программ соответствующего \textit{абстрактного sc-агента}.}
	\scnfileitem{Каждый такой \textit{scp-процесс}, соответствующий некоторой \textit{агентной scp-программе}, может быть связан с набором структур, описывающих блокировки различных типов. Таким образом, синхронизация взаимодействия параллельно выполняемых \textit{scp-процесcов} осуществляется так же, как и в случае любых других \textit{действий в sc-памяти}.}
	\scnfileitem{В рамках \textit{scp-процесса} могут создаваться дочерние \textit{scp-процессы}, однако синхронизация между ними при необходимости осуществляется посредством введения дополнительных внутренних блокировок. Таким образом, каждый \textit{scp-процесс} с точки зрения \textit{процессов в sc-памяти} является атомарным и законченным актом деятельности некоторого \textit{sc-агента}.}
	\scnfileitem{Во избежание нежелательных изменений в самом теле \textit{scp-процесса}, вся конструкция, сгенерированная на основе некоторой \textit{scp-программы} (весь текст \textit{scp-процесса}), должна быть добавлена в \textit{полную блокировку}, соответствующую данному \textit{scp-процессу}.}
	\scnfileitem{Все конструкции, сгенерированные в процессе выполнения \textit{scp-процесса}, автоматически попадают в \textit{полную блокировку}, соответствующую данному \textit{scp-процессу}. Дополнительно следует отметить, что знак самой этой структуры и вся метаинформация о ней также включаются в эту структуру.}
	\scnfileitem{При необходимости можно вручную разблокировать или заблокировать некоторую конструкцию каким-либо типом блокировки, используя соответствующие \textit{scp-операторы} класса \textit{scp-оператор управления блокировками}.}
	\scnfileitem{После завершения выполнения некоторого \textit{scp-процесса} его текст как правило, удаляется из \textit{\mbox{sc-памяти}}, а все заблокированные конструкции освобождаются (разрушаются знаки структур, обозначавших блокировки).}
	\scnfileitem{Несмотря на то, что каждый \textit{scp-оператор} представляет собой атомарное \textit{действие в sc-памяти}, дополнительные блокировки, соответствующие одному оператору не вводятся, чтобы избежать громоздкости и избытка дополнительных системных конструкций, создаваемых при выполнении некоторого \textit{scp-процесса}. Вместо этого используются блокировки, общие для всего \textit{scp-процесса}. Таким образом, агенты \textit{Абстрактной scp-машины} при интерпретации \textit{scp-операторов} работают только с учетом блокировок, общих для всего интерпретируемого \textit{scp-процесса}.}
	\scnfileitem{Как правило, частный \textit{класс действий}, соответствующий конкретной \textit{scp-программе} явно не вводится, а используется более общий класс \textit{scp-процесс}, за исключением тех случаев, когда введение специального \textit{класса действий} необходимо по каким-либо другим соображениям.}
	%<---TODO: check by human
\end{scnrelfromset}

\scnheader{scp-процесс}
\scntext{пояснение}{Под \textbf{\textit{scp-процессом}} понимается некоторое \textit{действие в sc-памяти}, однозначно описывающее конкретный акт выполнения некоторой \textit{scp-программы} для заданных исходных данных. Если \textit{scp-программа} описывает алгоритм решения какой-либо задачи в общем виде, то \textit{scp-процесс} обозначает конкретное действие, реализующее данный алгоритм для заданных входных параметро
	По сути, \textbf{\textit{scp-процесс}} представляет собой уникальную копию, созданную на основе \textit{scp-программы}, в которой каждой \textit{sc-переменной}, за исключением \textit{scp-переменных\scnrolesign}, соответствует сгенерированная \textit{sc-константа}.\\
	Принадлежность некоторого действия множеству \textit{scp-процессов} гарантирует тот факт, что в декомпозиции данного действия будут присутствовать только знаки элементарных действий (\textit{scp-операторов}), которые может интерпретировать реализация \textit{Абстрактной scp-машины}.}
\begin{scnrelfromvector}{пример выполнения}
	%TODO: check by human--->
	\scnitem{\scnfileimage[20em]{figures/sd_scp/process_example.png}}
		\begin{scnindent}
			\scntext{пояснение}{Осуществляется вызов \textit{scp-программы}. Генерируется соответствующий \textit{scp-процесс}. Происходит инициирование начального оператора scp-процесса \textit{Operator1}.} 
		\end{scnindent}
	\scnitem{\scnfileimage[20em]{figures/sd_scp/process_example2.png}}
		\begin{scnindent}	
			\scntext{пояснение}{Оператор \textit{Operator1} оказался безуспешно выполненным. Производится инициирование \textit{\mbox{scp-оператора} генерации трёхэлементной конструкции} \textit{Operator2}.}
		\end{scnindent}
	\scnitem{\scnfileimage[20em]{figures/sd_scp/process_example3.png}}
		\begin{scnindent}	
			\scntext{пояснение}{Оператор \textit{Operator2} выполнился. Производится инициирование \textit{scp-оператора завершения выполнения программы} \textit{Operator3}.}
		\end{scnindent}
	\scnitem{\scnfileimage[20em]{figures/sd_scp/process_example4.png}}
		\begin{scnindent}
			\scntext{пояснение}{Оператор \textit{Operator3} выполнился. Выполнение \textit{scp-процесса} завершается.}
		\end{scnindent}
%<---TODO: check by human

\end{scnrelfromvector}
\end{SCn}
