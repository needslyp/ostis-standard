\newpage\scnsegmentheader{Комплекс свойств, определяющих качество физической оболочки
    кибернетической системы}
\begin{scnsubstruct}
    \scnheader{качество физической оболочки кибернетической системы}
    \scnidtf{интегральное качество аппаратной (физической) основы кибернетической
        системы}
    \scnidtf{hardware кибернетической системы}
    \begin{scnrelfromlist}{свойство-предпосылка}

        \scnitem{качество памяти кибернетической системы}
        \scnitem{качество процессора кибернетической системы}
        \scnitem{качество сенсоров кибернетической системы}
        \scnitem{качество эффекторов кибернетической системы}
        \scnitem{приспособленность физической оболочки кибернетической системы к ее
            совершенствованию}
        \scnitem{удобство транспортировки кибернетической системы}
        \scnitem{надежность физической оболочки кибернетической системы}

    \end{scnrelfromlist}
    \scnheader{качество памяти кибернетической системы}
    \begin{scnreltolist}{свойство-предпосылка}

        \scnitem{качество информации, хранимой в памяти кибернетической системы}
        \scnitem{качество решателя задач кибернетической системы}

    \end{scnreltolist}
    \begin{scnrelfromlist}{свойство-предпосылка}

        \scnitem{способность памяти кибернетической системы обеспечить хранение
            высококачественной информации}
        \scnitem{способность памяти кибернетической системы обеспечить функционирование
            высококачественного решателя задач}
        \scnitem{объём памяти}

    \end{scnrelfromlist}
    \scnheader{память кибернетической системы}
    \scnidtf{компонент \textit{кибернетической системы}, представляющий собой
        внутреннюю среду \textit{кибернетической системы}, в которой она хранит
        (запоминает) и преобразует \textit{информационную модель} своей \textit{внешней
            среды}. При этом важно, чтобы память обеспечивала высокий уровень
        \textit{гибкости} указанной \textit{информационной модели}. Важно также, чтобы
        эта \textit{информационная модель} была моделью не только \textit{внешней
            среды} \bigskip \textit{кибернетической системы}, но также и моделью самой этой
        \textit{информационной модели} -- описанием её \textit{текущей ситуации},
        предыстории, закономерностей. Таким образом, \textit{кибернетическая система},
        имеющая \textit{память}, функционирует в двух средах -- во внешней, в которой
        существуют и преобразуются внешние(материальные) сущности, и во внутренней, в
        которой существуют и преобразуются(обрабатываются) внутренние
        \textit{информационные конструкции}.}
    \scntext{примечание}{\textit{Кибернетические системы}, находящиеся на низком уровне
        развития(качества) \textit{памяти} не имеют. Адаптационные механизмы такой
        кибернетической системы жестко запаяны в связях между блоками обработчика
        \textit{сигналов} при переходе от \textit{сигналов}, вырабатываемых
        \textit{сенсорами} к \textit{сигналам}, которые управляют
        \textit{эффекторами}.}\scnidtf{внутренняя среда кибернетической системы,
        обеспечивающая хранение и преобразование(обработку) информационной модели
        внешней среды кибернетической системы}
    \scntext{примечание}{Сам факт возникновения памяти в \textit{кибернетической системе}
        является важнейшим этапом её эволюции. Дальнейшее развитие \textit{памяти
            кибернетической системы}, обеспечивающее:\begin{scnitemize}

            \item хранение все более качественной информации, хранимой в памяти
            \item все более качественную организацию обработки этой информации, т.е.
            переход на поддержку(обеспечение) все более качественных моделей обработки
            информации\end{scnitemize}
        является важнейшим фактором эволюции \textit{кибернетических
            систем}.}

    \scnheader{способность памяти кибернетической системы обеспечить хранение высококачественной информации}
    \begin{scnrelfromlist}{свойство-предпосылка}

        \scnitem{способность системы обеспечить компактное хранение
            сложноструктурированных баз знаний}
            \begin{scnindent}
                \scntext{примечание}{Здесь имеется в виду необходимость перехода от
                    линейной организации, памяти на физическом уровне (как последовательности ячеек
                    памяти) к нелинейной, графодинамической памяти.}
            \end{scnindent}
        \scnitem{способность памяти кибернетической системы обеспечить хранение
            широкого многообразия знаний}
            \begin{scnindent}
                \scntext{примечание}{имеется в виду хранение гибридных баз знаний}
            \end{scnindent}
    \end{scnrelfromlist}

    \scnheader{способность памяти кибернетической системы обеспечить
        функционирование высококачественного решателя задач}
    \begin{scnrelfromlist}{свойство-предпосылка}

        \scnitem{качество доступа к информации, хранимой памяти кибернетической системы}
        \begin{scnindent}
            \scntext{примечание}{Здесь имеется в виду необходимость перехода от адресного к
                ассоциативному доступу, причем, с расширением многообразия видов реализуемых
                запросов, в частности, к реализации запросов фрагментов баз знаний по заданному
                образцу произвольного размера и произвольной конфигурации.}
        \end{scnindent}
        \scnitem{логико-семантическая гибкость памяти кибернетической системы}
        \scnitem{способность памяти кибернетической системы обеспечить интерпретацию
            широкого многообразия моделей решения задач}

    \end{scnrelfromlist}

    \scnheader{логико-семантическая гибкость памяти кибернетической системы}
    \scnidtf{степень близости физической организации памяти кибернетической системы
        к реализуемым ею базовым семантически целостным действиям над информацией,
        хранимой в памяти}
    \scnidtf{простота реализации базовых семантически целостных действий над
        информацией, хранимой в памяти кибернетической системы}
    \scntext{примечание}{Важен переход от мелких действий, к элементарным действиям,
        имеющим логико-семантический смысл (целостность, законченность)}
    
    \scnheader{качество процессора кибернетической системы}
    \scnrelto{свойство-предпосылка}{качество решателя задач кибернетической
        системы}
    \begin{scnrelfromlist}{свойство-предпосылка}

        \scnitem{способность процессора кибернетической системы обеспечить функционирования высококачественного решателя задач}
            \begin{scnrelfromlist}{свойство-предпосылка}

                \scnitem{многообразие моделей решения задач, интерпретируемых процессором
                    кибернетической системы}
                \scnitem{простота и качество интерпретации процессором системы широкого
                    многообразия моделей решения задач}
                \begin{scnindent}
                    \scntext{примечание}{Указанная простота определяется степенью близости
                        интерпретируемых моделей решения задач к физическому уровню организации
                        процессора кибернетической системы.}
                \end{scnindent}
                \scnitem{обеспечение процессором кибернетической системы качественного
                    управления информационными процессами в памяти}
                \begin{scnindent}
                    \scntext{примечание}{Речь идет о грамотном сочетание таких аспектов управление
                        процессами, как централизация и децентрализация, синхронность и асинхронность,
                        последовательность и параллельность.}\scnrelfrom{свойство-предпосылка}{уровень
                        параллелизма обработки информации в памяти кибернетической системы}
                    \scnidtf{максимальное количество одновременно выполняемых информационных
                        процессов в памяти кибернетической системы}
                \end{scnindent}
                \scnitem{быстродействие процессора кибернетической системы}

    \end{scnrelfromlist}
        

    \end{scnrelfromlist}
    \newpage\scnheader{многообразие моделей решения задач, интерпретируемых
        процессором кибернетической системы}
    \scntext{примечание}{Максимальным уровнем качества процессора кибернетической системы
        по данном параметру является его универсальность, т.е. его принципиальная
        возможность интерпретировать любую модель решения как интеллектуальных, так и
        неинтеллектуальных задач (алгоритмизацию, процедурную параллельную синхронную,
        процедруную параллельную асинхронную, продукционную, нейросетевую,
        генетическую, функциональную, целое семейство моделей).}
        
    \scnheader{качество сенсоров кибернетической системы}
    \scnrelfrom{свойство-предпосылка}{многообразие видов сенсоров кибернетической системы}
    \begin{scnindent}    
        \scnidtf{многообразие средств восприятия (отображения) информации о текущем
                состоянии внешней среды кибернетической системы и её собственной физической
                оболочки}
    \end{scnindent}
    

    \scnheader{качество эффекторов кибернетической системы}
    \scnrelfrom{свойство-предпосылка}{многообразие видов эффекторов кибернетической системы}
    \begin{scnindent}    
        \scnidtf{многообразие средств воздействия на собственную физическую оболочку
            кибернетической системы и через нее на внешнюю среду этой системы}
        \scntext{примечание}{Эффекторы кибернетической системы являются инструментами
            воздействия кибернетической системы на свою внешнюю среду.}
    \end{scnindent}

    \scnheader{приспособленность физической оболочки кибернетической системы к её
        совершенствованию}
    \scnidtf{приспособленность кибернетической системы к повышению качества её
        физической оболочки}
    \scnidtf{простота ремонта и совершенствования таких компонентов кибернетической
        системы как память, процессор, сенсоры, эффекторы}
    \scnrelfrom{частное свойство}{ремонтопригодность физической оболочки
        кибернетической системы}
    \begin{scnrelfromset}{группа свойств-предпосылок}

        \scnitem{гибкость физической оболочки кибернетической системы}
        \scnitem{стратифицированность физической оболочки кибернетической системы}
        \begin{scnindent}    
            \scnidtf{мобильность физической оболочки кибернетической системы}
            \scnidtf{легкость сохранения целостности физической оболочки кибернетической
                системы при внесении различных изменений (локализация области учета последствий
                внесения изменений, предсказуемость последствий)}
        \end{scnindent}

    \end{scnrelfromset}
    \bigskip\end{scnsubstruct}
    \scnsourcecomment{Завершили Сегмент “Комплекс свойств, определяющих качество физической оболочки кибернетической системы”}
