\begin{SCn}
    \scnsectionheader{Система ключевых знаков Cтандарта OSTIS}
    \begin{scnsubstruct}
        \scntext{пояснение}{\textit{Система ключевых знаков Стандарта OSTIS} должна
            стать целостным дополнением  к Оглавлению Стандарта OSTIS
            \begin{scnitemize}
                \item иерархия и последовательность ключевых знаков  должны четко
                соответствовать иерархии и последовательности разделов стандарта;
                \item система ключевых знаков Стандарта OSTIS, как и его Оглавление, должна
                восприниматься (читаться) как целостный понятный текст
            \end{scnitemize}
        }
        \scnheader{Стандарт OSTIS}
        \begin{scnrelfromvector}{ключевые знаки}
            \scnitem{база знаний ostis-системы}
                \begin{scnindent}
                    \scnidtf{база знаний, представленная в SC-коде}
                    \scnidtf{sc-модель базы знаний}
                \end{scnindent}
            \scnitem{технология проектирования баз знаний ostis-систем}
            \scnitem{логическая sc-модель обработки знаний}
            \scnitem{технология проектирования логических sc-моделей обработки знаний}
            \scnitem{продукционная sc-модель обработки знаний}
            \scnitem{технология проектирования продукционных sc-моделей обработки знаний}
            \scnitem{sc-модель искусственной нейронной сети}
            \scnitem{технология проектирования sc-моделей искусственных нейронных сетей}
            \scnitem{sc-модель интерфейса ostis-системы}
            \scnitem{технология проектирования  sc-моделей интерфейсов ostis-систем}
            \scnitem{sc-модель интерфейса ostis-системы}
                \begin{scnindent}
                    \scnidtf{онтологическая модель интерфейса, построенная на основе SC-кода}
                \end{scnindent}
            \scnitem{технология проектирования sc-моделей интерфейсов ostis-систем}
            \scnitem{программная платформа реализации ostis-систем, построенная на основе
                системы управления графовыми базами данных}
                \begin{scnindent}
                    \scnidtf{программная система интерпретации логико-семантических моделей
                        ostis-систем, построенная на основе графовой СУБД}
                    \scnidtf{система управления базами знаний (СУБЗ) ostis-систем, построенная на
                        основе графовых СУБД}
                \end{scnindent}
            \scnitem{ассоциативный семантический компьютер для ostis-систем}
                \begin{scnindent}
                    \scnidtf{компьютер с ассоциативной графодинамической (структурно
                        реконфигурируемой) памятью, ориентированный на реализацию ostis-систем}
                    \scnidtf{компьютер с ассоциативной графодинамической памятью, обеспечивающий
                        интерпретацию логико-семантических моделей ostis-систем}
                    \scnidtf{аппаратная платформа реализации ostis-систем}
                \end{scnindent}
            \scnitem{Проект OSTIS}
            \scnitem{Стандарт OSTIS}
            \scnitem{Экосистема OSTIS}
                \begin{scnindent}
                    \scnidtf{Экосистема ostis-систем и их пользователей}
                    \scnidtf{Вариант построения smart-общества (общества 5.0) на основе
                        ostis-систем}
                \end{scnindent}
            \scnitem{агентно-ориентированная модель обработки информации}
                \begin{scnindent}
                    \scnidtf{многоагентная модель обработки информации}
                    \scnidtf{\textit{модель обработки информации}, рассматривающая \textit{процесс
                        обработки информации} как \textit{деятельность}, выполняемую некоторым
                        \textit{коллективом} самостоятельных \textit{информационных агентов} (агентов
                        обработки информации)}
                \end{scnindent}
        \end{scnrelfromvector}
        \scnrelfrom{ключевой объект спецификации}{Технология OSTIS}
        \scnrelfrom{основные создаваемые продукты}{ostis-система}
            \begin{scnindent}
                \scnidtf{множество всевозможных ostis-систем}
                \scnidtf{компьютерная система, построенная по Технологии OSTIS}
            \end{scnindent}
        \begin{scnrelfromvector}{ключевые понятия, соответствующие принципам, лежащим в
                основе}
            \scnitem{cмысловое представление информации}
            \scnitem{агентно-ориентированная обработка информации}
            \scnitem{интерфейс компьютерной системы}
                \begin{scnindent}
                    \scnidtf{интерфейс компьютерной системы, построенный на основе онтологий}
                    \scnidtf{ontology based interface}
                \end{scnindent}
            \scnitem{мультимодальность}
            \scnitem{конвергенция}
            \scnitem{семантическая совместимость}
            \scnitem{унификация}
            \scnitem{мультимодальная база знаний}
            \scnitem{универсальный язык смыслового представления знаний}
            \scnitem{мультимодальный решатель задач}
            \scnitem{мультимодальный интерфейс компьютерной системы}
            \scnitem{гибридная интеллектуальная компьютерная система}
                \begin{scnindent}
                    \scnidtf{мультимодальная интеллектуальная компьютерная система}
                \end{scnindent}    
            \scnitem{мультимодальный (гибридный) характер и.к.с. в целом}
            \scnitem{мультимодальный характер баз знаний и.к.с.}
            \scnitem{мультимодальный решатель задач и.к.с.}
            \scnitem{мультимодальный (мультиязычный) интерфейс и.к.с.}
            \scnitem{конвергенция и.к.с.(знаний, моделей, решателей задач, моделей
                взаимодействий с внешней средой, моделей общения с внешним субъектом)}
            \scnitem{семантическая совместимость и.к.с.(знаний, моделей, решателей задач,
                моделей взаимодействий с внешней средой, моделей общения с внешним субъектом)}
            \scnitem{онтологическая модель}
            \scnitem{онтологическая логико-семантическая модель и.к.с.}
            \scnitem{онтологическая модель базы знаний и.к.с.}
            \scnitem{онтологическая модель решатель задач и.к.с.}
            \scnitem{онтологическая модель интерфейса  и.к.с.}
            \scnitem{неатомарный раздел}
            \scnitem{атомарный раздел}
            \scnitem{ключевой знак*}
                \begin{scnindent}
                    \scnidtf{ключевая сущность*}
                \end{scnindent}
            \scnitem{SC-код}
            \scnitem{SCg-код}
            \scnitem{SCs-код}
            \scnitem{SCn-код}
            \scnitem{декомпозиция*}
            \scnitem{конкатенация*}
            \scnitem{предметная область}
                \begin{scnindent}
                    \scnidtf{sc-модель предметной области}
                    \scnidtf{sc-текст, являющийся представлением предметной области}
                \end{scnindent}
            \scnitem{максимальный класс объекта исследования\scnrolesign}
            \scnitem{немаксимальный класс объекта исследования\scnrolesign}
                \begin{scnindent}
                    \scnidtf{подкласс максимального объекта исследования\scnrolesign}
                \end{scnindent}
            \scnitem{исследуемое отношение\scnrolesign}
            \scnitem{исследуемый параметр\scnrolesign}
            \scnitem{онтология}
            \scnitem{алфавит*(языка)}
            \scnitem{сформированное множество}
                \begin{scnindent}
                    \scnidtf{конечное множество, все элементы которого представлены
                    соответствующими sc-элементами}
                \end{scnindent}
            \scnitem{бинарное отношение}
            \scnitem{ориентированное отношение}
            \scnitem{первый домен*}
            \scnitem{второй домен*}
            \scnitem{пояснение*}
            \scnitem{семантическая эквивалентность*}
            \scnitem{следствие*}
            \scnitem{примечание*}
            \scnitem{определение*}
        \end{scnrelfromvector}
    \end{scnsubstruct}
\end{SCn}