\begin{SCn}
    \scnsegmentheader{Уточнение понятия навыка, понятия класса методов и понятия модели решения задач}
    \begin{scnsubstruct}
        \scniselement{сегмент базы знаний}
        
        \scnheader{навык}
        \scnidtf{умение}
        \scnidtf{объединение \textit{метода} с его исчерпывающей спецификацией --- \textit{полным представлением операционной семантики метода}}
        \scnidtf{метод, интерпретация (выполнение, использование) которого полностью может быть осуществлено данной кибернетической системой, в памяти которой указанный метод хранится}
        \scnidtf{метод, который данная кибернетическая система умеет (может) применять}
        \scnidtf{метод + метод его интерпретации}
        \scnidtf{умение решать соответствующий класс эквивалентных задач}
        \scnidtf{метод плюс его операционная семантика, описывающая то, как интерпретируется (выполняется, реализуется) этот метод, и являющаяся одновременно операционной семантикой соответствующей модели решения задач}
        \begin{scnsubdividing}
            \scnitem{активный навык}
            \begin{scnindent}
                \scnidtf{самоинициирующийся навык}
            \end{scnindent}
            \scnitem{пассивный навык}
        \end{scnsubdividing}
        \scntext{пояснение}{\textit{Навыки} могут быть \textit{пассивными навыками}, то есть такими \textit{навыками}, применение которых должно явно инициироваться каким-либо агентом, либо \textit{активными навыками}, которые инициируются самостоятельно при возникновении соответствующей ситуации в базе знаний. Для этого в состав \textit{активного навыка} помимо \textit{метода} и его операционной семантики включается также \textit{sc-агент}, который реагирует на появление соответствующей ситуации в базе знаний и инициирует интерпретацию \textit{метода} данного \textit{навыка}.\\
            Такое разделение позволяет реализовать и комбинировать различные подходы к решению задач, в частности, \textit{пассивные навыки} можно рассматривать в качестве способа реализации концепции интеллектуального пакета программ.}
        
        \scnheader{класс методов}
        \scnrelto{семейство подклассов}{метод}
        \scnidtf{множество методов, для которых можно \uline{унифицировать} представление (спецификацию) этих методов}
        \scnidtf{множество всевозможных методов решения задач, имеющих общий язык представления этих методов}
        \scnidtf{множество всевозможных методов, представленных на данном языке}
        \scnidtf{множество методов, для которых задан язык представления этих методов}
        \scnhaselement{процедурный метод решения задач}
        \begin{scnindent}
        	\scnsuperset{алгоритмический метод решения задач}
        \end{scnindent}
        \scnhaselement{логический метод решения задач}
        \begin{scnindent}
        	\scnsuperset{продукционный метод решения задач}
        	\scnsuperset{функциональный метод решения задач}
        \end{scnindent}
        \scnhaselement{искусственная нейронная сеть}
        \begin{scnindent}
        	\scnidtf{класс методов решения задач на основе искусственных нейронных сетей}
        \end{scnindent}
        \scnhaselement{генетический алгоритм}
        \scnidtf{множество методов основанных на общей онтологии}
        \scnidtf{множество методов, представленных на одинаковом языке}
        \scnidtf{множество методов решений задач, которому соответсвует специальный язык (например, sc-язык), обеспечивающий представление методов из этого множества}
        \scnidtf{множество методов, которому ставится в соответствие отдельная модель решения задач}
        
        \scnheader{язык представления методов}
        \scnidtf{язык методов}
        \scnidtf{язык представления методов, соответствующих определенному классу методов}
        \begin{scnindent}
        	\scntext{примечание}{Таких специализированных языков может быть выделено целое множество, каждому из которых будет соответствовать своя модель решения задач (т.е. свой интерпретатор)}
        \end{scnindent}
        \scnidtf{язык (например sc-язык) представлений методов соответствующего класса методов}
        \scnsubset{язык}
        \scnidtf{язык программирования}
        \scnsuperset{язык представления методов обработки информации}
        \begin{scnindent}
	        \scnidtf{язык программирования внутренних действий кибернетической системы, выполняемых в их памяти}
	        \scnidtf{язык представления методов решения задач в памяти кибернетических систем}
	    \end{scnindent}
        \scnsuperset{язык представления методов решения задач во внешней среде кибернетических систем}
        \begin{scnindent}
        	\scnidtf{язык программирования внешних действий кибернетических систем}
        \end{scnindent}
        
        \scnheader{модель решения задач}
        \scnidtf{метаметод интерпретации соответствующего класса методов}
        \scnsubset{метод}
        \scnidtf{метаметод}
        \scnidtf{абстрактная машина интерпретации соответствующего класса методов}
        \scnidtf{иерархическая система микропрограмм, обеспечивающих интерпретацию соответствующего класса методов}
        \scnsuperset{алгоритмическая модель решения задач}
        \scnsuperset{процедурная параллельная синхронная модель решения задач}
        \scnsuperset{процедурная параллельная асинхронная модель решения задач}
        \scnsuperset{продукционная модель решения задач}
        \scnsuperset{функциональная модель решения задач}
        \scnsuperset{логическая модель решения задач}
        \begin{scnindent}
	        \scnsuperset{четкая логическая модель решения задач}
	        \scnsuperset{нечеткая логическая модель решения задач}
	    \end{scnindent}
        \scnsuperset{нейросетевая модель решения задач}
        \scnsuperset{генетическая модель решения задач}
        \scntext{примечание}{Для интерпретации \uline{всех} моделей решения задач может быть использован агентно-ориентированный подход}
        \scntext{пояснение}{Каждая \textit{модель решения задач} задается:
            \begin{scnitemize}
                \item соответствующим классом методов решения задач, т.е. языком представления методов этого класса;
                \item предметной областью этого класса методов;
                \item онтологией этого класса методов (т.е. денотационной семантикой языка представления этих методов);
                \item операционной семантикой указанного класса методов.
            \end{scnitemize}}
        
        \scnheader{модель решения задач*}
        \scneq{сужение отношения по первому домену(спецификация*; класс методов)*}
        \scnidtf{спецификация \textit{класса методов}*}
        \scnidtf{спецификация \textit{языка представления методов}*}
        \begin{scnreltoset}{обобщенное объединение}
            \scnitem{синтаксис языка представления методов соответствующего класса*}
            \scnitem{денотационная семантика языка представления методов соответствующего класса*}
            \scnitem{операционная семантика языка представления методов соответствующего класса*}
        \end{scnreltoset}
        \scntext{примечание}{Каждому конкретному \textit{классу методов} взаимно однозначно соответствует \textit{язык представления методов}, принадлежащих этому (специфицируемому) \textit{классу методов}. Таким образом, спецификация каждого \textit{класса методов} сводится к спецификации соответствующего \textit{языка представления методов}, т.е. к описанию его синтаксической, денотационной семантики и операционной семантики.Примерами \textit{языков представления методов} являются все \textit{языки программирования}, которые в основном относятся к подклассу \textit{языков представления методов} --- к \textit{языкам представления методов обработки информации}. Но сейчас все большую актуальность приобретает необходимость создания эффективных формальных языков представления методов выполнения действий во внешней среде кибернетических систем. Без этого комплексная автоматизация, в частности, в промышленной сфере невозможна.}
        
        \scnheader{денотационная семантика языка представления методов соответствующего класса}
        \scnrelto{второй домен}{денотационная семантика языка представления методов соответствующего класса*}
        \scnidtf{онтология соответствующего класса методов}
        \scnidtf{денотационная семантика соответствующего класса методов}
        \scnidtf{денотационная семантика языка (sc-языка), обеспечивающего представление методов соответствующего класса}
        \scnidtf{денотационная семантика соответствующей модели решения задач}
        \scntext{примечание}{если речь идет о языке, обеспечивающем внутреннее представление методов соответствующего класса в ostis-системе, то синтаксис этого языка совпадает с синтаксисом SC-кода}
        \scnsubset{онтология}
        \scnheader{операционная семантика языка представления методов соответствующего класса}
        \scnrelto{второй домен}{операционная семантика языка представления методов соответствующего класса*}
        \scnidtf{метаметод интерпретации соответствующего класса методов}
        \scnidtf{семейство агентов, обеспечивающих интерпретацию (использования) любого метода, принадлежащего соответствующему классу методов}
        \scnidtf{операционная семантика соответствующей модели решения задач}
        
        \scnheader{язык представления обобщенных формулировок задач для различных классов задач}
        \scntext{примечание}{Поскольку каждому \textit{методу} соответствует \textit{обобщенная формулировка задач}, решаемых с помощью этого \textit{метода}, то каждому \textit{классу методов} должен соответствовать не только определенный \textit{язык представления методов}, принадлежащих указанному \textit{классу методов}, но и определенный \textit{язык представления обобщенных формулировок задач для различных классов задач}, решаемых с помощью \textit{методов}, принадлежащих указанному \textit{классу методов}.}
        \bigskip
    \end{scnsubstruct}
    
    \scnendsegmentcomment{Уточнение понятия навыка, понятия класса методов и понятия модели решения задач}
    \scnsegmentheader{Уточнение понятия деятельности, понятия вида деятельности и понятия технологии}
    \begin{scnsubstruct}
        \scniselement{сегмент базы знаний}
        \scnheader{деятельность}
        \scnidtftext{пояснение}{сложный процесс, состоящий из действий, направленных на достижение нескольких \uline{разных} целей (т.е. целей, не связанных отношением цель-подцель). При этом некоторые из указанных максимальных целей могут достигаться с помощью одного и того же метода или одного и того же (фиксированного) семейства методов}
        \scnsuperset{физическая деятельность}
        \begin{scnindent}
        	\scnidtf{деятельность по преобразованию материальных сущностей (физических объектов)}
        \end{scnindent}
        \scnsuperset{информационная деятельность}
        \begin{scnindent}
	        \scnidtf{деятельность, направленная на обработку информацию}
	        \scnidtf{умственная деятельность}
        	\scntext{примечание}{Информационная деятельность является необходимым компонентом физической деятельности, обеспечивающим принятие решений, планирование и управление физическим процессом}
        \end{scnindent}
        \scnsubset{процесс}
        \scnidtf{целостный, целенаправленный процесс \uline{поведения} (функционирования) одного субъекта или сообщества субъектов, осуществляемый на основе хорошо или не очень хорошо продуманной и согласованной \textit{технологии} в последнем случае качество деятельности определяется уровнем интеллекта единоличного или коллективного субъекта, осуществляющего этот целенаправленный процесс.}
        \scnidtf{система действий, являющаяся некоторым кластером семантически близких действий, обладающих семантической близостью, семантической связностью и семантической целостностью}
        \scnidtf{трудно выполнимая семантически целостная система действий}
        \scnidtf{кластер множества действий, определяемый семантической близостью этих действий}
        \scnidtf{система связанных между собой действий, имеющих общий контекст, общую область выполнения этих действий}
        \scntext{примечание}{В состав каждой конкретной \textit{деятельности} входят \textit{действия}, являющиеся \textit{поддействиями}* других \textit{действий}, входящих в состав этой же \textit{деятельности}. При этом для каждого \textit{действия}, входящего в состав \textit{деятельности}, все поддействия этого \textit{действия} также входят в состав этой \textit{деятельности}.\\
        В состав каждой конкретной \textit{деятельности} входят также \textit{действия}, не являющиеся \textit{поддействиями}* других \textit{действий}, входящих в состав этой же \textit{деятельности}. Такие первичные (независимые, самостоятельные, автономные) \textit{действия} для заданной \textit{деятельности} могут инициироваться \uline{извне} этой \textit{деятельности} с помощью соответствующих инициирующих эти \textit{действия ситуаций} или \textit{событий}. Примерами таких инициирующих ситуаций, порождающих соответствующие действия, являются:
            \begin{scnitemize}
                \item появление в \textit{базе знаний} каких-либо противоречий, информационных дыр, информационного мусора;
                \item появление в \textit{базе знаний} описаний (информационных моделей) каких-либо нештатных ситуаций в сложном объекте управления, на которые необходимо реагировать;
                \item появление в \textit{базе знаний} формулировок различного рода задач с явным указанием инициирования соответствующих действий, направленных на решение этих задач.
            \end{scnitemize}
            К числу указанных первичных (независимых) \textit{действий}, входящих в состав \textit{объединенной деятельности кибернетической системы}, также относятся:
            \begin{scnitemize}
                \item сложное действие, целью которого является перманентное обеспечение комплексной \textit{безопасности кибернетической системы};
                \item сложное действие, целью которого является перманентное повышение качества информации (базы знаний), хранимой в памяти \textit{кибернетической системы};
                \item сложное действие, целью которого является перманентное повышение \textit{качества решателя задач кибернетической системы};
                \item сложное действие, целью которого является перманентная поддержка высокого уровня \textit{семантической совместимости} кибернетической системы со своими партнерами.
            \end{scnitemize}}
        
        \scnheader{отношение, заданное на множестве*(деятельность)}
        \scnhaselement{субъект*}
        \begin{scnindent}
	        \scnidtf{быть субъектом заданного действия или деятельности*}
	        \scnidtf{кибернетическая система, которая в рамках заданного действия или деятельности выполняет ту или иную роль, воздействует на некий объект действия, используя тот или иной инструмент*}
        	\scniselement{отношение, заданное на множестве*(действие)}
        \end{scnindent}
        \scnhaselement{контекст*}
        \begin{scnindent}
	        \scnidtf{информационный контекст, в рамках которого осуществляется выполнение заданного действия или деятельности*}
	        \scnidtf{область исполнения действия или деятельности*}
	        \scnidtf{область действия или деятельности*}
	        \scnrelfrom{первый домен}{(действие $\cup$ деятельность)}
	        \scnidtf{совокупность знаний, достаточных для информационного обеспечения заданного действия или заданной деятельности}
	        \scniselement{отношение, заданное на множестве* (действие)}
	        \scntext{примечание}{Локализация (минимизация) \textit{контекста} заданного действия или деятельности является важнейшим подготовленным этапом, обеспечивающим существенное снижение накладных расходов при непосредственном выполнении этого \textit{действия} или \textit{деятельности}.}
	   \end{scnindent}
	   \scntext{примечание}{Чаще всего \textit{контекстом} заданного \textit{действия} или \textit{деятельности} является некоторая \textit{предметная область} вместе с соответствующей ей интегрированной (объединенной) \textit{онтологией}. Поэтому хорошо продуманная декомпозиция \textit{базы знаний} интеллектуальной компьютерной системы на иерархическую систему \textit{предметных областей} и соответствующих им \textit{онтологий} имеет важное практическое значение, существенно повышающее качество (в частности, быстродействие) \textit{решателя задач} интеллектуальной компьютерной системы благодаря априорному  разбиению множества выполняемых \textit{действий} (решаемых задач) по соответствующих им \textit{контекстам}.}
        
        \scnheader{следует отличать*}
        \begin{scnhassubset}
                \scnitem{действие}
                \begin{scnindent}
	                \scnidtf{процесс достижения конкретной цели конкретных обстоятельствах}
	                \scnidtf{процесс решения конкретной задачи в конкретных условиях}
	                \scnidtf{процесс задуманный, инициированный и осуществленный некоторым (или некоторыми) субъектами (кибернетическими системами)}
	                \scntext{примечание}{\textit{действие} (точнее, соответствующая форма участия в его выполнении) является частью (фрагментом) \textit{деятельности} всех участвующих в этом субъектов (кибернетических систем)}
                \end{scnindent}
                \scnitem{деятельность}
                \begin{scnindent}
	                \scnidtf{система действий выполняемых соответствующим субъектом (кибернетической системой) скрепленное общим контекстом и определенным набором используемых навыков и инструментов}
	                \scntext{примечание}{В отличие от \textit{действия}, \textit{деятельность} носит чаще всего перманентный характер в рамках времени существования соответствующего субъекта}
                \end{scnindent}
        \end{scnhassubset}
        
        \scnheader{деятельность кибернетической системы}
        \scnidtf{полная система действий, выполняемых соответствующей кибернетической системой}
        \scnidtf{деятельность субъекта}
        \scnidtf{система всех действий соответствующего субъекта}
        \begin{scnsubdividing}
            \scnitem{внутренняя деятельность субъекта}
            \begin{scnindent}
                \scnidtf{внутренняя деятельность соответствующего субъекта}
                \scnidtf{деятельность некоторого субъекта по обработке информации}
                \scnidtf{информационная деятельность}
            \end{scnindent}
            \scnitem{поведение субъекта}
            \begin{scnindent}
                \scnidtf{внешнее поведение соответствующего субъекта}
                \scnidtf{деятельность субъекта во внешней среде}
            \end{scnindent}
        \end{scnsubdividing}
        
        \scnheader{вид деятельности}
        \scnrelto{семейство подклассов}{деятельность}
        \scnidtf{класс семантически целостных систем действия, для которых можно унифицировать используемые методы, информационные ресурсы и инструменты}
        \scnidtf{класс трудно выполнимых и семантически целостных систем сложных действий}
        \scnidtf{класс кластеров систем действий}
        \scnidtf{множество деятельностей, которые могут быть реализованы с помощью общей технологии}
        \scnhaselement{устранение противоречий в базе знаний}
        \scnhaselement{устранение информационных дыр в базе знаний}
        \scnhaselement{ликвидация информационного мусора в базе знаний}
        \scnhaselement{управление сложным внешним объектом}
        \scnhaselement{поддержка семантической совместимости с партнерами}
        \scnhaselement{проектирование}
        \begin{scnindent}
	        \scnidtf{проектная деятельность}
	        \scnidtf{построение такого описания (в частности, описания структуры) некоторого материального объекта, которого достаточно для воспроизводства (реализации, материализации) этого объекта либо при одиночном (уникальном), либо при массовом (промышленном) воспроизводстве указанного объекта}
	        \scntext{примечание}{Примерами проектирования являются:
	            \begin{scnitemize}
	                \item проектирование здания;
	                \item проектирование машиностроительной конструкции;
	                \item проектирование микросхемы;
	                \item проектирование ostis-системы;
	                \item разработка системы шунтирования сердца;
	                \item разработка такого описания сложной геометрической фигуры, которого было бы достаточно для построения изображения (рисунка) этой фигуры с помощью, например, циркуля и линейки.
	            \end{scnitemize}}
        \end{scnindent}
        \scnhaselement{разработка плана производства материального объекта по заданному проекту этого объекта}
        \begin{scnindent}
	        \scnsuperset{разработка плана единичной реализации материального объекта по заданному проекту этого объекта}
	        \scnsuperset{разработка плана массовой реализации материальных объектов по заданному их типовому проекту}
	        \scntext{примечание}{Примерами данного вида действий являются:
	            \begin{scnitemize}
	                \item разработка плана-графика строительства конкретного здания;
	                \item разработка типового плана строительства зданий по заданному их типовому проекту;
	                \item разработка типового плана операций шунтирования сердца;
	                \item разработка алгоритма построения \uline{изображения} заданной геометрической фигуры с помощью циркуля и линейки
	            \end{scnitemize}}
        \end{scnindent}
        \scnhaselement{производство}
        \begin{scnindent}
        	\scnidtf{воспроизводство материального объекта по заданному его проекту и плану реализации}
	        \scnidtf{производственная деятельность}
	        \scntext{примечание}{Примерами данного вида действий являются:
	            \begin{scnitemize}
	                \item непосредственно строительство конкретного здания;
	                \item проведение конкретной хирургической операции;
	                \item процесс построения \uline{изображения} (рисунка) геометрической фигуры с помощью циркуля и линейки.
	            \end{scnitemize}}
        \end{scnindent}
        \scnhaselement{реинжиниринг}
        \scnhaselement{анализ}
        \scnhaselement{интеграция}
        \begin{scnindent}
        	\scnidtf{синтез}
        \end{scnindent}
        \scnhaselement{деятельность в области здравоохранения}
        \scnhaselement{образовательная деятельность}
        \scnhaselement{эксплуатация сложного объекта}
        \scnhaselement{научно-исследовательская деятельность}
        \scnhaselement{управление}
        \begin{scnindent}
        	\scnsuperset{целенаправленная координация деятельности нескольких субъектов}
        	\begin{scnindent}
        		\scnidtf{управление целенаправленной коллективной деятельностью нескольких субъектов}
        	\end{scnindent}
        \end{scnindent}
        
        \scnheader{проектирование}
        \scnidtf{действие, направленное на построение (разработку) такой \uline{информационной} модели (проекта) некоторой \uline{материальной} сущности, которой \uline{достаточно}, чтобы соответствующий индивидуальный или коллективный субъект по соответствующей технологии (т.е. с помощью соответствующих методов и средств (инструментов)) смог воспроизвести (изготовить) указанную материальную сущность либо в одном экземпляре, либо в достаточно большом количестве таких экземпляров (копий), т.е. воспроизвести в промышленном масштабе}
        
        \scnheader{производство}
        \scnidtf{воспроизводство}
        \scnidtf{изготовление}
        \scnidtf{реализация}
        \scnidtf{материализация}
        \scnidtf{построение, синтез материальной сущности (артефакта)}
        \scnidtf{изготовление материальной сущности в одной или во множестве экземпляров (копий)}
        \scnidtf{производство (как действие)}
        
        \scnheader{реинжиниринг}
        \scnidtf{модификация}
        \scnidtf{внесение изменений в некую сущность}
        \scnidtf{обновление}
        \scnidtf{реинжиниринг}
        \scnidtf{перепроектирование}
        \scnidtf{реконфигурация}
        \scnidtf{трансформирование}
        \scnsuperset{совершенствование}
        \begin{scnindent}
	        \scnidtf{модификация, направленной на повышение качества модифицируемой сущности}
	        \scnidtf{повышение качества}
	        \scnidtf{улучшение}
	        \scnsuperset{самосовершенствование}
	        \begin{scnindent}
	        	\scnidtf{совершенствование, выполняемое самой совершенствуемой сущностью}
	        \end{scnindent}
	        \scnsuperset{совершенствование, осуществляемое извне}
	        \scntext{примечание}{Самосовершенствоваться и обучаться могут только достаточно развитые кибернетические системы. Но совершенствоваться усилиями внешних субъектов могут любые сущности.}
	    \end{scnindent}
        
        \scnheader{анализ}
        \scnidtf{построение (разработка, создание) спецификации (описания) основных связей и/или структуры, свойств, закономерностей, соответствующих (описываемой) сущности}
        \scntext{примечание}{Объектом анализа может быть не только материальная сущность, но и процесс, ситуация, статическая структура, внешняя информационная конструкция, знание, понятие и другие абстрактные сущности}
        \scnheader{интеграция}
        \scnidtf{синтез}
        \scnidtf{соединение}
        \scnidtf{объединение}
        \scnidtf{сборка}
        \begin{scnsubdividing}
            \scnitem{эклектичная интеграция}
            \begin{scnindent}
                \scnidtf{интеграция без разрушения целостности интегрируемых сущностей}
                \scnidtf{интеграция без взаимопроникновения}
                \scnidtf{соединение систем по их входам/выходам}
            \end{scnindent}
            \scnitem{глубокая интеграция}
            \begin{scnindent}
                \scnidtf{интеграция, в результате которой получается гибридная сущность}
                \scnidtf{интеграция с разрушением целостности (взаимопроникновением диффузий) интегрируемых сущностей}
                \scnidtf{бесшовная интеграция}
            \end{scnindent}
        \end{scnsubdividing}
        
        \scnheader{вид деятельности}
        \scntext{пояснение}{Если классу легко выполнимых сложных действий ставится в соответствие чаще всего \uline{один} \textit{метод} и, возможно, некоторый набор инструментальных средств, используемых в этом методе, то каждому виду деятельности ставится в соответствие своя \textbf{\textit{технология}}, включающая в себя некоторый набор используемых \textit{методов}, а также набор \textit{инструментальных средств}, используемых в этих \textit{методах}. Сложность здесь заключается:
            \begin{scnitemize}
                \item в нетривиальности организации использования всего арсенала имеющейся \textit{технологии} для реализации (выполнения) каждой соответствующей \textit{деятельности};
                \item в трудности, а часто и в принципиальной невозможности \uline{полностью} автоматизировать реализацию соответствующей \textit{деятельности}.
            \end{scnitemize}}
        
        \scnheader{следует отличать*}
        \begin{scnhaselementset}
            \scnitem{действие}
            \begin{scnindent}
                \scnhaselementrole{пример}{Процесс доказательства Теоремы Пифагора}
                \begin{scnindent}
                	\scniselement{действие направленное на построение доказательства теоремы Геометрии Евклида}
                \end{scnindent}
            \end{scnindent}
            \scnitem{класс действий}
            \begin{scnindent}
                \scnhaselementrole{пример}{процесс доказательства теоремы}
                \begin{scnindent}
	                \scnidtftext{имя нарицательное}{действие, направленное на построение доказательства (логического обоснования) теоремы}
	                \scnidtftext{имя собственное}{Класс действий, направленных на построение доказательств (логических обоснований) всевозможных теорем в различных формальных теориях}
	            \end{scnindent}
            \end{scnindent}
            \scnitem{деятельность}
            \begin{scnindent}
                \scnhaselementrole{пример}{Процесс эволюции Геометрии Евклида}
                \begin{scnindent}
	                \scnidtf{Процесс эволюции формальной теории, являющейся формальным представлением Геометрии Евклида}
	                \scntext{пояснение}{В данный процесс входит и генерация гипотез в рамках Геометрии Евклида, и доказательство теорем, и выявление противоречий между высказываниями, и разрешение этих противоречий, и минимизация числа используемых определяемых понятий, и многое другое}
	            \end{scnindent}
                \scntext{примечание}{\textit{деятельность} --- это то, что превращает множество самостоятельных и в определенной степени независимых \textit{действий}, принадлежащих разным \textit{классам действий}, в целостную, целенаправленную, сбалансированную систему \textit{действий}, ориентированную, прежде всего на поддержание качества и эволюцию \textit{кибернетических систем}, а также на обеспечение их адаптации к новым, ранее не предусмотренным обстоятельствам.}
            \end{scnindent}
            \scnitem{вид деятельности}
            \begin{scnindent}
                \scnhaselementrole{пример}{процесс эволюции формальной теории}
                \begin{scnindent}
                	\scnidtftext{имя собственное}{Класс процессов, направленных на эволюцию всевозможных формальных теорий (логических онтологий), которая также включает в себя возможность коррекции этих теорий.}
                \end{scnindent}
            \end{scnindent}
        \end{scnhaselementset}
        
        \scnheader{сужение отношения по первому домену(спецификация*; вид деятельности)*}
        \scnidtftext{часто используемый sc-идентификатор}{спецификация вида деятельности*}
        \scneq{технология*}
        \begin{scnindent}
	        \scnidtf{технология реализации (выполнения) деятельности соответствующего (заданного) вида*}
	        \scnrelfrom{второй домен}{\textbf{технология}}
	        \begin{scnindent}
	            \scnidtf{технология соответствующего вида деятельности}
	            \scnrelboth{аналог}{декларативный метод выполнения действий соответствующего класса}
	            \begin{scnindent}	
	            	\scnrelboth{аналог}{декларативная спецификация выполнения действия}
	            \end{scnindent}
	            \scntext{пояснение}{\textit{технология} (как спецификация соответствующего вида деятельности) включает в себя:
	            \begin{scnitemize}
	                    \item указание \textit{контекста}* специфицируемого \textit{вида деятельности};
	                    \item указание \textit{множества используемых методов}*, множества используемых инструментов, а также используемых материалов.
	                \end{scnitemize}}
	        \end{scnindent}
        \end{scnindent}
        
        \scnheader{технология}
        \scntext{пояснение}{Каждая \textit{технология} представляет собой комплекс \textit{методов} (методик) и средств, обеспечивающих выполнение некоторого множества \textit{действий}, входящих в состав соответствующего \textit{вида деятельности}. Каждая \textit{технология} задается:
            \begin{scnitemize}
                \item множеством методов (методик), которое разбивается на классы методов, эквивалентных по своей операционной семантике (по набору агентов, осуществляющих интерпретацию соответствующего класса методов);
                \item множеством агентов, являющихся средством интерпретации методов из указанного выше множества.
            \end{scnitemize}
            Указанное множество агентов также разбивается на подмножества, каждое из которых соответствует своему классу методов и обеспечивает интерпретацию методов только этого класса.}
        \scnidtf{множество (комплекс) навыков, обеспечивающих выполнение такого множества действий (задач), для которых отсутствует общий метод их выполнения}
        \scnidtf{методика, инструментарий и дополнительные ресурсы, которые обеспечивают выполнение каждой конкретной деятельности, принадлежащей соответствующему виду деятельности}
        \scntext{пояснение}{с формальной точки зрения каждая технология задается ориентированной связкой, компонентами которой являются
            \begin{scnitemize}
                \item знак множества используемых методов
                \item знак множества используемых инструментов
                \item знак множества дополнительных используемых ресурсов
            \end{scnitemize}}
        \scnidtf{комплекс методов и средств (инструментов), с помощью которого некий субъект (который может быть как индивидуальным, так и коллективным) осуществляет некоторую деятельность (некоторое целенаправленное множество действий, входящих в состав этой деятельности)}
        \scnsuperset{технология научно-теоретической деятельности}
        \scnsuperset{технология проектирования}
        \begin{scnindent}
	        \scnidtf{технология проектной деятельности}
	        \scnidtf{технология построения такой информационной модели соответствующей сущности (артефакта), которой достаточно для воспроизводства этой сущности}
	    \end{scnindent}
        \scnsuperset{технология производства}
        \begin{scnindent}
	        \scnidtf{технология производственной деятельности}
	        \scnidtf{технология воспроизводства некоторого вида сущностей по заданным проектам этих сущностей}
	    \end{scnindent}
        \scnsuperset{технология здравоохранения}
        \scnsuperset{технология образования}
        \begin{scnindent}
	        \scnidtf{технология подготовки молодых специалистов}
	        \scnidtf{технология образовательной деятельности}
	    \end{scnindent}
	    	
        \scnheader{отношение, заданное на множестве* (технология*)}
        \scnhaselement{методы*}
        \begin{scnindent}
        	\scnidtf{семейство методов, используемых в специфицируемой технологии с дополнительным указанием их иерархии (т.е. с указанием того, какие методы используются при реализации других методов)}
        \end{scnindent}
        \scnhaselement{активные инструменты*}
        \begin{scnindent}
	        \scnidtf{средства, которые сами способны выполнять некоторые действия, но при этом ими надо как-то управлять (например, транспортные средства, компьютеры, )}
	        \scnidtf{средства автоматизации*}
	    \end{scnindent}
        \scnhaselement{пассивные инструменты*}
        \begin{scnindent}
       		\scnidtf{средства, которые сами ничего делать не могут (например, молоток, лопата, ножницы)}
        \end{scnindent}
        \scnhaselement{комплектация*}
        \scnhaselement{расходные средства*}
        \scnhaselement{сырье*}
        \scnhaselement{продукты*}
        \scnhaselement{общий продукт*}
        \begin{scnindent}
        	\scnidtf{объединенный (интегрированный) продукт*}
        \end{scnindent}
        \scnhaselement{реализация технологии*}
        \begin{scnindent}
        	\scnidtf{вариант (форма) реализации технологии*}
        \end{scnindent}
        \scnhaselement{частная технология*}
        \begin{scnindent}
        	\scnidtf{быть частной технологией по отношению к заданной технологии*}
        \end{scnindent}
        
        \scnheader{продукты*}
        \scnidtf{производимые сущности*}
        \scnidtf{изготавливаемые материальные сущности*}
        \scnidtf{продукция*}
        \scnidtf{результаты выполнения соответствующего множества действий, осуществляемых во внешней среде*}
        \scnidtf{продукты технологии*}
        \scnidtf{множество материальных сущностей, производимых (создаваемых, порождаемых, изготавливаемых) с помощью заданной технологии*}
        \scnidtf{то, что является сухим остатком при использовании данной технологии*}
        
        \scnheader{технология}
        \scntext{примечание}{Поскольку разработка каждой конкретной \textit{технологии} требует больших затрат, очень важно, чтобы \textit{технологии} создавались не под конкретные \textit{деятельности}, а для целых классов деятельностей (\textit{видов деятельности}). При этом важно, чтобы разрабатываемые \textit{технологии} охватывали как можно большее количество деятельностей, входящих в состав указанных \textit{видов деятельности}. Из этого следует целесообразность конвергенции и унификации различных сфер \textit{деятельности} для того, чтобы повысить мощность применения (использования) каждой разрабатываемой \textit{технологии}. Кроме того важна \textit{совместимость технологий}, позволяющая решать \textit{задачи}, требующие одновременного использования нескольких \textit{технологий}, причем, в непредсказуемых сочетаниях. Очень важно также, кроме \textit{видов деятельности}, которым соответствуют конкретные \textit{технологии}, ввести \textit{обобщенные виды деятельности} и построить их иерархии явно фиксировать стандарты, которым должны соответствовать все виды соответствующего обобщенного \textit{вида деятельности}. Это необходимо для обеспечения совместимости \textit{технологий}. Все используемые технологии должны пронизывать друг друга и составлять стройную иерархическую систему совместимых технологий (сумму технологий).}
        
        \scnheader{класс технологий}
        \scnidtf{множество похожих технологий, использующих, например, одинаковые методики и/или одинаковые активные инструменты и/или одинаковые пассивные инструменты и/или похожие множества продуктов}
        \scnhaselement{технология проектирования}
        \begin{scnindent}
	        \scnsuperset{технология проектирования интеллектуальных компьютерных систем}
	        \scnsuperset{технология проектирования программных систем}
	        \scnsuperset{технология проектирования микросхем}
        	\scnsuperset{технология машиностроительного проектирования}
        \end{scnindent}
        \scnhaselement{технология рецептурного производства}
        \begin{scnindent}
	        \scnsuperset{технология производства молочных продуктов}
	        \scnsuperset{технология производства мясных продуктов}
	        \scnsuperset{технология фармацевтического производства}
	    \end{scnindent}
        \bigskip
    \end{scnsubstruct}
    \scnendsegmentcomment{Уточнение понятия деятельности, понятия вида деятельности и понятия технологии}
\end{SCn}
